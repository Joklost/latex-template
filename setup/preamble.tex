\documentclass[11pt,twoside,a4paper,openright]{report}
%%%%%%%%%%%%%%%%%%%%%%%%%%%%%%%%%%%%%%%%%%%%%%%%
% Language, Encoding and Fonts
% http://en.wikibooks.org/wiki/LaTeX/Internationalization
%%%%%%%%%%%%%%%%%%%%%%%%%%%%%%%%%%%%%%%%%%%%%%%%
% Select encoding of your inputs. Depends on
% your operating system and its default input
% encoding. Typically, you should use
%   Linux  : utf8 (most modern Linux distributions)
%            latin1
%   Windows: ansinew
%            latin1 (works in most cases)
%   Mac    : applemac
% Notice that you can manually change the input
% encoding of your files by selecting "save as"
% an select the desired input encoding.
\usepackage{scrextend}
\usepackage[utf8]{inputenc}
%\usepackage{float}
\usepackage{floatrow}
\usepackage{booktabs}
\usepackage[table,xcdraw]{xcolor}
% Make latex understand and use the typographic
% rules of the language used in the document.
\usepackage[english]{babel}
% Use the vector font Latin Modern which is going
% to be the default font in latex in the future.
\usepackage{lmodern}
% Choose the font encoding
\usepackage[T1]{fontenc}
\usepackage{lastpage} % Tilføjet af Jesper, får sidetal til at virke.
%%%%%%%%%%%%%%%%%%%%%%%%%%%%%%%%%%%%%%%%%%%%%%%%
% Graphics and Tables
% http://en.wikibooks.org/wiki/LaTeX/Importing_Graphics
% http://en.wikibooks.org/wiki/LaTeX/Tables
% http://en.wikibooks.org/wiki/LaTeX/Colors
%%%%%%%%%%%%%%%%%%%%%%%%%%%%%%%%%%%%%%%%%%%%%%%%
% load a colour package
\usepackage[table]{xcolor}

\definecolor{aaublue}{RGB}{33,26,82}% dark blue
\definecolor{editgreen}{HTML}{008000}% dark green
% The standard graphics inclusion package
\usepackage{graphicx}
% Set up how figure and table captions are displayed
\usepackage[within=none]{caption}
\captionsetup{%
  font=footnotesize,% set font size to footnotesize
  labelfont=bf % bold label (e.g., Figure 3.2) font
}
% Make the standard latex tables look so much better
\usepackage{array,booktabs}
% Enable the use of frames around, e.g., theorems
% The framed package is used in the example environment
\usepackage{framed}

\usepackage{enumitem}
%%%%%%%%%%%%%%%%%%%%%%%%%%%%%%%%%%%%%%%%%%%%%%%%
% Mathematics
% http://en.wikibooks.org/wiki/LaTeX/Mathematics
%%%%%%%%%%%%%%%%%%%%%%%%%%%%%%%%%%%%%%%%%%%%%%%%
% Defines new environments such as equation,
% align and split
\usepackage{amsmath}
% Adds new math symbols
\usepackage{amssymb}
% Use theorems in your document
% The ntheorem package is also used for the example environment
% When using thmmarks, amsmath must be an option as well. Otherwise \eqref doesn't work anymore.
\usepackage[framed,amsmath,thmmarks]{ntheorem}

%%%%%%%%%%%%%%%%%%%%%%%%%%%%%%%%%%%%%%%%%%%%%%%%
% Page Layout
% http://en.wikibooks.org/wiki/LaTeX/Page_Layout
%%%%%%%%%%%%%%%%%%%%%%%%%%%%%%%%%%%%%%%%%%%%%%%%
% Change margins, papersize, etc of the document
\usepackage[top=1in, bottom=1.5in, left=1in, right=1in]{geometry}
\setlength\parindent{0pt}
%\usepackage[
%  inner=28mm,% left margin on an odd page
%  outer=41mm,% right margin on an odd page
%  ]{geometry}
% Modify how \chapter, \section, etc. look
% The titlesec package is very configureable
\usepackage{titlesec}
\titleformat*{\section}{\normalfont\Large\bfseries\color{aaublue}}
\titleformat*{\subsection}{\normalfont\large\bfseries\color{aaublue}}
\titleformat*{\subsubsection}{\normalfont\normalsize\bfseries\color{aaublue}}
%\titleformat*{\paragraph}{\normalfont\normalsize\bfseries\color{aaublue}}
%\titleformat*{\subparagraph}{\normalfont\normalsize\bfseries\color{aaublue}}

% Clear empty pages between chapters
\let\origdoublepage\cleardoublepage
\newcommand{\clearemptydoublepage}{%
  \clearpage
  {\pagestyle{empty}\origdoublepage}%
}
\let\cleardoublepage\clearemptydoublepage

% Change the headers and footers
\usepackage{fancyhdr}
\pagestyle{fancy}
\fancyhf{} %delete everything
\renewcommand{\headrulewidth}{0pt} %remove the horizontal line in the header
\fancyhead[RE]{\color{aaublue}\small\nouppercase\leftmark} %even page - chapter title
\fancyhead[LO]{\color{aaublue}\small\nouppercase\rightmark} %uneven page - section title
\fancyhead[LE,RO]{\thepage} %page number on all pages
\setlength{\headheight}{14pt}
% Do not stretch the content of a page. Instead,
% insert white space at the bottom of the page
\raggedbottom
% Enable arithmetics with length. Useful when
% typesetting the layout.
\usepackage{calc}

%%%%%%%%%%%%%%%%%%%%%%%%%%%%%%%%%%%%%%%%%%%%%%%%
% Bibliography
% http://en.wikibooks.org/wiki/LaTeX/Bibliography_Management
%%%%%%%%%%%%%%%%%%%%%%%%%%%%%%%%%%%%%%%%%%%%%%%%
% Add the \citep{key} command which display a
% reference as [author, year]
\usepackage[numbers]{natbib}
% Appearance of the bibliography
\bibliographystyle{plain} %plain, unsrtnat

%%%%%%%%%%%%%%%%%%%%%%%%%%%%%%%%%%%%%%%%%%%%%%%%
% Misc
%%%%%%%%%%%%%%%%%%%%%%%%%%%%%%%%%%%%%%%%%%%%%%%%
% Add bibliography and index to the table of
% contents
\usepackage[nottoc]{tocbibind}
% Add the command \pageref{LastPage} which refers to the
% page number of the last page
\usepackage[
%  disable, %turn off todonotes
  colorinlistoftodos, %enable a coloured square in the list of todos
  textwidth=\marginparwidth, %set the width of the todonotes
  textsize=scriptsize, %size of the text in the todonotes
  ]{todonotes}

\usepackage[T1]{fontenc}
\usepackage{inconsolata}

\usepackage{color}
\definecolor{bluekeywords}{rgb}{0.13,0.13,1}
\definecolor{greencomments}{rgb}{0,0.5,0}
\definecolor{redstrings}{rgb}{0.9,0,0}

\usepackage{listings}
%\lstset{language=[Sharp]C,
%  showspaces=false,
%  showtabs=false,
%  breaklines=true,
%  showstringspaces=false,
%  breakatwhitespace=true,
%  escapeinside={(*@}{@*)},
%  commentstyle=\color{greencomments},
%  keywordstyle=\color{bluekeywords},
%  stringstyle=\color{redstrings},
%  basicstyle=\ttfamily
%}
% \lstdefinelanguage{CSharp}
% {
% sensitive=true,
% morekeywords=[1]{
% abstract, as, base, break, case,
% catch, checked, class, const, continue,
% default, delegate, do, else, enum,
% event, explicit, extern, false,
% finally, fixed, for, foreach, goto, if,
% implicit, in, interface, internal, is,
% lock, namespace, new, null, operator,
% out, override, params, private,
% protected, public, readonly, ref,
% return, sealed, sizeof, stackalloc,
% static, struct, switch, this, throw,
% true, try, typeof, unchecked, unsafe,
% using, virtual, volatile, while, bool,
% byte, char, decimal, double, float,
% int, lock, object, sbyte, short, string,
% uint, ulong, ushort, void},
% morecomment=[l]{//},
% morecomment=[s]{/*}{*/},
% morecomment=[l][keywordstyle4]{\#},
% morestring=[b]",
% morestring=[b]',
% }
% \lstset{
% backgroundcolor=\color[rgb]{0.95, 0.95, 0.95},
% tabsize=2,
% rulecolor=,
% basicstyle=\scriptsize,
% upquote=true,
% aboveskip={1.5\baselineskip},
% columns=fixed,
% showstringspaces=false,
% extendedchars=true,
% breaklines=true,
% prebreak = \raisebox{0ex}[0ex][0ex]{\ensuremath{\hookleftarrow}},
% frame=single,
% showtabs=false,
% showspaces=false,
% showstringspaces=false,
% identifierstyle=\ttfamily,
% keywordstyle=\color[rgb]{1.0,0,0},
% keywordstyle=[1]\color[rgb]{0,0,0.75},
% keywordstyle=[2]\color[rgb]{0.5,0.0,0.0},
% keywordstyle=[3]\color[rgb]{0.127,0.427,0.514},
% keywordstyle=[4]\color[rgb]{0.4,0.4,0.4},
% commentstyle=\color[rgb]{0.133,0.545,0.133},
% stringstyle=\color[rgb]{0.639,0.082,0.082},
% }

\usepackage{courier}
\definecolor{gray}{rgb}{0.4,0.4,0.4}
\definecolor{darkblue}{rgb}{0.0,0.0,0.6}
\definecolor{cyan}{rgb}{0.0,0.6,0.6}

\lstset{
  basicstyle=\ttm,
  columns=fullflexible,
  showstringspaces=false,
  commentstyle=\color{gray}\upshape
}

\lstdefinelanguage{XML}
{
  morestring=[b]",
  morestring=[s]{>}{<},
  morecomment=[s]{<?}{?>},
  stringstyle=\color{black},
  identifierstyle=\color{darkblue},
  keywordstyle=\color{cyan},
  morekeywords={xmlns,version,type}% list your attributes here
}

%\setmonofont{Consolas} %to be used with XeLaTeX or LuaLaTeX
\definecolor{bluekeywords}{rgb}{0,0,1}
\definecolor{greencomments}{rgb}{0,0.5,0}
\definecolor{redstrings}{rgb}{0.64,0.08,0.08}
\definecolor{xmlcomments}{rgb}{0.5,0.5,0.5}
\definecolor{types}{rgb}{0.17,0.57,0.68}

\lstset{language=[Sharp]C,
captionpos=b,
%numbers=left, %Nummerierung
%numberstyle=\tiny, % kleine Zeilennummern
frame=lines, % Oberhalb und unterhalb des Listings ist eine Linie
showspaces=false,
showtabs=false,
breaklines=true,
showstringspaces=false,
breakatwhitespace=true,
escapeinside={(*@}{@*)},
commentstyle=\color{greencomments},
morekeywords={partial, var, value, get, set},
keywordstyle=\color{bluekeywords},
stringstyle=\color{redstrings},
basicstyle=\ttfamily\small,
}

\renewcommand{\lstlistingname}{Code}% Listing -> Algorithm
\renewcommand{\lstlistlistingname}{List of \Code s}% List of Listings -> List of Algorithms
%\definecolor{lightgray}{rgb}{.9, .9, .9}
%\definecolor{darkgray}{rgb}{.4, .4, .4}
%\definecolor{purple}{rgb}{0.65, 0.12, 0.82}
%
%\lstdefinelanguage{JavaScript}{
%        keywords={break, case, catch, continue, debugger, default, delete, do, else, finally, for, function, if, in, instanceof, new, return, switch, this, throw, try, typeof, var, void, while, with},
%        keywordstyle=\color{blue}\bfseries,
%        ndkeywords={class, export, boolean, throw, implements, import, this},
%        ndkeywordstyle=\color{darkgray}\bfseries,
%        identifierstyle=\color{black},
%        sensitive=false,
%        comment=[l]{//},
%        morecomment=[s]{/*}{*/},
%        commentstyle=\color{purple}\ttfamily,
%        stringstyle=\color{red}\ttfamily,
%        morestring=[b]',
%        morestring=[b]"}
%
%\lstset{
%        language=JavaScript,
%        backgroundcolor=\color{lightgray},
%        extendedchars=true,
%        basicstyle=\footnotesize\ttfamily,
%        showstringspaces=false,
%        showspaces=false,
%        numbers=left,
%        numberstyle=\footnotesize\ttfamily,
%        numbersep=9pt,
%        tabsize=2,
%        breaklines=true,
%        showtabs=false,
%        frame=leftline,
%        caption=\lstname,
%        literate={\$}{{\textcolor{blue}{\$}}}1
%        }

%%%%%%%%%%%%%%%%%%%%%%%%%%%%%%%%%%%%%%%%%%%%%%%%
% Hyperlinks
% http://en.wikibooks.org/wiki/LaTeX/Hyperlinks
%%%%%%%%%%%%%%%%%%%%%%%%%%%%%%%%%%%%%%%%%%%%%%%%
% Enable hyperlinks and insert info into the pdf
% file. Hypperref should be loaded as one of the
% last packages
\usepackage{hyperref}
\hypersetup{%
	plainpages=false,%
	pdfauthor={SW701E17 - Tobias Morell, Jonas Kloster Jacobsen, Marius Lauge Nørgaard, Mark Holst, Malte Rosenbjerg Andersen og Jesper Windelborg Nielsen},%
	pdftitle={ThingsOfInterest},%
	pdfsubject={Project Report},%
	bookmarksnumbered=true,%
	colorlinks,%
	citecolor=aaublue,%
	filecolor=aaublue,%
	linkcolor=aaublue,% you should probably change this to black before printing
	urlcolor=aaublue,%
	pdfstartview=FitH%
}
% Gets code to look okay -> And looks like code
\usepackage{listings}
\lstset{
	xleftmargin = -1.5cm,
	numbers = left,
	numbersep = -35pt
}
% Quotations package
\usepackage[autostyle]{csquotes}
% PDF package
\usepackage[final]{pdfpages}
\usepackage[acronym,toc]{glossaries}
\usepackage{longtable}
\usepackage{subcaption}
\addtokomafont{labelinglabel}{\sffamily\bfseries}
\usepackage{dirtytalk}

\usepackage{silence}
\WarningFilter{glossaries}{Deprecated command}% Removes warning starting with "Deprecated command"
\WarningFilter{glossaries}{Empty glossary}