% see, e.g., http://en.wikibooks.org/wiki/LaTeX/Customizing_LaTeX#New_commands
% for more information on how to create macros

%%%%%%%%%%%%%%%%%%%%%%%%%%%%%%%%%%%%%%%%%%%%%%%%
% Macros for the titlepage
%%%%%%%%%%%%%%%%%%%%%%%%%%%%%%%%%%%%%%%%%%%%%%%%
%Creates the aau titlepage
\newcommand{\aautitlepage}[3]{%
  {
    %set up various length
    \ifx\titlepageleftcolumnwidth\undefined
      \newlength{\titlepageleftcolumnwidth}
      \newlength{\titlepagerightcolumnwidth}
    \fi
    \setlength{\titlepageleftcolumnwidth}{0.5\textwidth-\tabcolsep}
    \setlength{\titlepagerightcolumnwidth}{\textwidth-2\tabcolsep-\titlepageleftcolumnwidth}
    %create title page
    \thispagestyle{empty}
    \noindent%
    \begin{tabular}{@{}ll@{}}
      \parbox{\titlepageleftcolumnwidth}{
        \iflanguage{danish}{%
          \includegraphics[width=\titlepageleftcolumnwidth]{figures/aau_logo_da}
        }{%
          \includegraphics[width=\titlepageleftcolumnwidth]{figures/aau_logo_en}
        }
      } &
      \parbox{\titlepagerightcolumnwidth}{\raggedleft\sf\small
        #2
      }\bigskip\\
       #1 &
      \parbox[t]{\titlepagerightcolumnwidth}{%
      \textbf{Abstract:}\bigskip\par
        \fbox{\parbox{\titlepagerightcolumnwidth-2\fboxsep-2\fboxrule}{%
          #3
        }}
      }\\
    \end{tabular}
    \vfill
    \iflanguage{danish}{%
      \noindent{\footnotesize\emph{Rapportens indhold er frit tilgængeligt, men offentliggørelse (med kildeangivelse) må kun ske efter aftale med forfatterne.}}
    }{%
      \noindent{\footnotesize\emph{The content of this report is freely available, but publication (with reference) may only be pursued due to agreement with the author.}}
    }
    \clearpage
  }
}

%Create english project info
\newcommand{\englishprojectinfo}[8]{%
  \parbox[t]{\titlepageleftcolumnwidth}{
    \textbf{Title:}\\ #1\bigskip\par
    \textbf{Theme:}\\ #2\bigskip\par
    \textbf{Project Period:}\\ #3\bigskip\par
    \textbf{Project Group:}\\ #4\bigskip\par
    \textbf{Participant(s):}\\ #5\bigskip\par
    \textbf{Supervisor(s):}\\ #6\bigskip\par
    \textbf{Copies:} #7\bigskip\par
    \textbf{Page Numbers:} \pageref{LastPage}\bigskip\par
    \textbf{Date of Completion:}\\ #8
  }
}

%Create danish project info
\newcommand{\danishprojectinfo}[8]{%
  \parbox[t]{\titlepageleftcolumnwidth}{
    \textbf{Titel:}\\ #1\bigskip\par
    \textbf{Tema:}\\ #2\bigskip\par
    \textbf{Projektperiode:}\\ #3\bigskip\par
    \textbf{Projektgruppe:}\\ #4\bigskip\par
    \textbf{Deltagere:}\\ #5\bigskip\par
    \textbf{Vejleder:}\\ #6\bigskip\par
    \textbf{Oplagstal:} #7\bigskip\par
    \textbf{Sidetal:} \pageref{LastPage}\bigskip\par
    \textbf{Afleveringsdato:}\\ #8
  }
}

%%%%%%%%%%%%%%%%%%%%%%%%%%%%%%%%%%%%%%%%%%%%%%%%
% An example environment
%%%%%%%%%%%%%%%%%%%%%%%%%%%%%%%%%%%%%%%%%%%%%%%%
\theoremheaderfont{\normalfont\bfseries}
\theorembodyfont{\normalfont}
\theoremstyle{break}
\def\theoremframecommand{{\color{aaublue!50}\vrule width 5pt \hspace{5pt}}}
\newshadedtheorem{exa}{Example}[chapter]
\newenvironment{example}[1]{%
		\begin{exa}[#1]
}{%
		\end{exa}
}

% Table macro.
% 1: Table Latex
% 2: Caption
% 3: Label
\definecolor{lightgray}{gray}{0.9}
\newcommand{\makeTable}[3]{
    \rowcolors{1}{}{lightgray}
    \begin{table}[H]
        \centering
            \begin{tabular}
            #1
            \end{tabular}
        \caption{#2}
        \label{table:#3}
    \end{table}
    \rowcolors{0}{}{}
}

% Table macro.
% 1: Table Latex
% 2: Caption
% 3: Label
\newcommand{\makeTablePB}[3]{
    \rowcolors{1}{}{lightgray}
    \begin{longtable}
        #1
        \caption{#2}
        \label{table:#3}
    \end{longtable}
    \rowcolors{0}{}{}
}

\newcommand{\tableref}[1]{Table \ref{table:#1}}
\newcommand{\figureref}[1]{Figure \ref{figure:#1}}
\newcommand{\coderef}[1]{Code \ref{code:#1}}
\newcommand{\secref}[1]{Section \ref{sec:#1}}
\newcommand{\chapref}[1]{Chapter \ref{chap:#1}}
\newcommand{\appendixref}[1]{Appendix \ref{app:#1}}
\newcommand{\lineref}[1]{Line \ref{line:#1}}

%\newcommand{\acrfull}[1]{\acrlong{#1} (\acrshort{#1})}

\newcommand{\dquot}[1]{``{#1}''}
\newcommand{\squot}[1]{`{#1}'}

\newcommand{\storycard}[8]{
\begin{minipage}[h]{.5\textwidth}
    \fbox{\parbox{\textwidth}{
        \textbf{Story Number:} #1\smallbreak
        \textbf{Author:} #2\smallbreak
        \textbf{Priority:} #3\smallbreak
        \textbf{Date:} #4\smallbreak
        \smallbreak
        \textbf{As a} #5\smallbreak
        \textbf{I want to} #6\smallbreak
        \textbf{So that} #7\smallbreak
        \smallbreak
        \textbf{Considerations:}\smallbreak
        #8
    }}
\end{minipage}
}